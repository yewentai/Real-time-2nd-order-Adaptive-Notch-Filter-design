\title{ANF: ANF-LMS algorithm }


\begin{equation}
 s(m)=y(m)+\rho(m) a(m-1) s(m-1)-\rho^{2}(m) s(m-2) \\
 e(m)=s(m)-a(m-1) s(m-1)+s(m-2)
\end{equation}

\begin{itemize}
  \item Adaptation strategy: only FIR portion of filter is adapted, coefficients are then copied to IIR portion of filter
  \item ANF-LMS filter coefficient update:
\end{itemize}

\begin{equation}
a(m) = a(m-1)+\mu\left[-\frac{\partial e^{2}(m)}{\partial a(m-1)}\right] \\
 =a(m-1)+\mu\left[-2 e(m) \frac{\partial e(m)}{\partial a(m-1)}\right] \\ =a(m-1)+\mu[2 e(m) s(m-1)]
\end{equation}


\begin{itemize}
  \item $2^{\text {nd }}$ order ANF-LMS algorithm
\end{itemize}

Algorithm $1: 2^{\text {nd }}$ order ANF-LMS algorithm

Require: step size $\mu$, initial pole radius $\rho(-1)$, final pole radius $\rho(\infty)$, exponential decay time constant $\lambda$, input data $\{y(m)\}_{m=0}^{N-1}$, initial conditions $s(-1), s(-2), a(-1)$

Ensure: $2^{\text {nd }}$ order ANF parameter $\{a(m)\}_{m=0}^{N-1}$

1: for $m=0, \ldots, N-1$ do

2: $\quad \rho(m)=\lambda \rho(m-1)+(1-\lambda) \rho(\infty)$

3: $\quad s(m)=y(m)+\rho(m) a(m-1) s(m-1)-\rho^{2}(m) s(m-2)$

4: $\quad e(m)=s(m)-a(m-1) s(m-1)+s(m-2)$

5: $\quad a(m)=a(m-1)+2 \mu e(m) s(m-1)$

6: end for

\section{what is direct from I and II}

In digital signal processing, Direct Form I and Direct Form II refer to two different structures used to implement digital filters, particularly Infinite Impulse Response (IIR) filters. From the document "Digital Signal Processing-2: Adaptive Notch Filters for Acoustic Feedback Control", there's an implementation of an Adaptive Notch Filter (ANF) in Direct Form II, which is detailed below:

\subsection{Direct Form II Implementation:}

Equation: The ANF is implemented using the following difference equations:

$x(t)=y(t)+\rho(t) a(t-1) x(t-1)-\rho 2(t) x(t-2)$

$e(t)=x(t)-a(t-I) x(t-I)+x(t-2)$

\paragraph{Variables:}

$y(t)$ : Input to the ANF.

$\mathrm{e}(\mathrm{t})$ : Output from the ANF.

$x(t)$ : An auxiliary variable used in the filter.

$a(t)$ : The filter coefficient that appears in both the numerator and denominator of the transfer function.

$\rho(t)$ : A parameter that controls the radius of the poles in the filter.

Filter Coefficient Update: The FIR portion of the filter is adapted to track the frequency of narrow-band components, and then the coefficients are copied to the IIR portion of the filter. This is an approximation suggested by Travassos-Romano et al..

\subsection{Understanding Direct Form I and II:}

\paragraph{Direct Form I:} This form represents the filter using a straightforward translation of the difference equation. It uses two distinct sections for implementing the feedforward (zeroes) and feedback (poles) parts of the filter.

\paragraph{Direct Form II:} This structure is a more efficient version, especially in terms of memory usage. It combines the feedforward and feedback sections into a single delay line, reducing the number of required memory locations.

\section{To implement an adaptive notch filter using the ANF-LMS algorithm. according to the paper, the zero and pole should lie on the same radial line, correct?}

Yes, you are correct. According to the document "Digital Signal Processing-2: Adaptive Notch Filters for Acoustic Feedback Control," Rao et al. proposed a constraint for an Infinite Impulse Response (IIR) filter, specifically stating that the poles and zeros should lie on the same radial lines within the unit circle. In this configuration, the zeros lie between the poles and the unit circle. This constraint is integral to the design of the adaptive notch filter using the ANF-LMS (Least Mean Squares) algorithm.

\subsection{Key Points of the Constraint:}

\begin{center}
\includegraphics[max width=\textwidth]{images/2023_12_24_70b29604aa91f5b1dce9g-2}
\end{center}

This arrangement where zeros and poles are aligned on the same radial lines is crucial for the performance of the adaptive notch filter, as it enables the filter to precisely attenuate specific frequencies while maintaining stability and control over the filter's response characteristics.

Implementing an adaptive notch filter using the ANF-LMS algorithm with this constraint in mind will help in effectively creating a filter that can adaptively suppress narrow-band interference or noise while preserving other parts of the signal spectrum.

\section{what will be the possible bode plot? can you give me an instance and interpret it?}

\subsection{Hypothetical Bode Plot Instance:}

\paragraph{Magnitude Plot:}

At the Notch Frequency $\left(w_{i}\right)$ : A sharp dip (notch) in magnitude will be observed due to the zero $\left(z_{i}\right)$ close to the unit circle. This represents significant attenuation at the notch frequency.

Around the Notch Frequency: Just before and after the notch frequency, there might be a slight increase in magnitude due to the presence of the pole $\left(p_{i}\right)$, indicating a resonance effect.

Other Frequencies: For frequencies far from the notch frequency, the magnitude response will be relatively flat, indicating minimal or no attenuation.

\paragraph{Phase Plot:}

At and Around the Notch Frequency: You will observe a rapid phase shift. This is typical around the frequencies where zeros and poles are located, as they induce phase changes in the filter's response.

Other Frequencies: The phase response will be more gradual and consistent across frequencies that are not near the notch.

\paragraph{Interpretation:}

The magnitude dip at the notch frequency effectively filters out a specific frequency component (or a narrow band of frequencies) from the input signal. This is especially useful in scenarios where you need to suppress a persistent, narrow-band interference (like a tone or hum) without affecting the rest of the signal spectrum.

The resonance effect near the notch frequency, indicated by the slight peak in magnitude just before and after the notch, is a characteristic of having poles close to the unit circle. This needs to be controlled to avoid amplifying frequencies too much, which could lead to instability. The phase response of the filter is also crucial, especially in applications where the phase of the output signal relative to the input is important.

\section{in our assignment, to implement a narrowband adaptive notch filter (ANF), the ri, radial line, for zero, should be on the unit circle, and hence $I$.}
In the context of implementing a narrowband Adaptive Notch Filter (ANF), positioning the radial line for the zero $\left(r_{i}\right)$ on the unit circle (i.e., setting $r_{i}=1$ ) is a critical design choice. This positioning ensures that the zero is at the edge of the unit circle in the z-plane, which corresponds to the highest level of attenuation at the specific notch frequency. Here's how this impacts the filter's behavior:

\subsection{Implications of Setting $\left(r_{i}=1\right)$ for Zero:}

\paragraph{Maximum Attenuation at Notch Frequency:} By placing the zero on the unit circle $\left(r_{i}=1\right)$ the filter achieves the maximum possible attenuation at the notch frequency. This is ideal for filtering out a very narrow band of frequencies - the filter essentially "notches out" the unwanted frequency.

\paragraph{Filter Selectivity: }This positioning enhances the selectivity of the filter. It makes the filter highly effective in targeting and eliminating a specific narrow frequency band without affecting the adjacent frequencies. Stability Considerations: While zeros can be on the unit circle without stability issues, care must be taken to ensure that the poles of the filter remain inside the unit circle for the system to be stable.

\paragraph{Practical Considerations:} In practical implementations, slight deviations from $\left(r_{i}=1\right)$ might be considered to account for non-idealities in the system, such as quantization errors or to avoid excessively sharp filtering which might affect adjacent frequencies.

\paragraph{Adaptive Aspect:} Since it's an adaptive filter, the parameters, including the position of the zero, can dynamically adjust in response to changes in the signal or the interference, making the filter versatile in varying signal conditions.

In summary, placing the zero of the ANF on the unit circle is a strategic decision in filter design to achieve strong attenuation at a specific narrow frequency band while maintaining overall system stability and filter effectiveness.

\section{what is Q-factor?}

The Q-factor, or Quality factor, is a parameter that characterizes the behavior of a resonator, filter, or oscillator in terms of its bandwidth relative to its center frequency. It's a key concept in various fields, including electrical engineering, acoustics, and physics. Here's a more detailed explanation:

\begin{center}
\includegraphics[max width=\textwidth]{images/2023_12_24_70b29604aa91f5b1dce9g-3}
\end{center}

\subsection{Implications and Applications:}

\paragraph{Narrow vs. Wide Bandwidth:}

A high Q-factor indicates a narrow bandwidth relative to the center frequency. This means the system is more selective, resonating or responding strongly to a narrow range of frequencies.

A low Q-factor implies a wider bandwidth, with the system responding over a broader range of frequencies.

\paragraph{Resonators and Filters:}

In resonators or filters, a high Q-factor is desirable when a narrow frequency response is needed, such as in notch filters or band-pass filters for specific frequencies.

Conversely, applications that require a wider frequency response, like in broadband antennas, would prefer a lower Q-factor.

\paragraph{Energy Losses:}

The Q-factor also relates to the rate of energy loss relative to the stored energy in the system. A higher $\mathrm{Q}$-factor indicates lower energy losses and thus a slower rate of energy dissipation.

\paragraph{Oscillators:}

In oscillators, the Q-factor influences the stability and purity of the oscillation frequency. A higher Qfactor can lead to more stable and less noisy oscillations.

\paragraph{Physical Systems:}

In mechanical and acoustic systems, the Q-factor describes how underdamped a system is. For instance, a tuning fork or a bell has a high Q-factor, resonating for a long time at a specific frequency.

In the context of an Adaptive Notch Filter (ANF), the Q-factor would be significant in determining how sharply the filter can isolate and attenuate a specific narrowband frequency. A higher Q-factor in this case means the filter is more effective at targeting a very narrow frequency range.

\section{What's the Q-factor used in the ANF python code?}
To calculate the Q-factor of the Adaptive Notch Filter (ANF) implemented in your code, we need to understand the characteristics of the filter, especially its center frequency and bandwidth. However, the Q-factor cannot be directly derived from the given code without additional information about the filter's response, particularly the bandwidth at the notch frequency.

In your code, the ANF is designed to remove sinusoidal interferences at frequencies $400 \mathrm{~Hz}$ and $1200 \mathrm{~Hz}$ from a signal sampled at $8000 \mathrm{~Hz}$. The rho parameter, set to 0.8 , affects the radius of the poles of the filter in the z-plane and thus influences the bandwidth and the sharpness of the notch.

\subsection{Understanding Q-Factor in Context of Your ANF:}

\begin{center}
\includegraphics[max width=\textwidth]{images/2023_12_24_70b29604aa91f5b1dce9g-4(1)}
\end{center}

\subsection{Practical Approach:}

\paragraph{Simulation:} You could simulate the filter's frequency response using a tool like MATLAB or Python. By plotting the magnitude response, you can measure the bandwidth at $-3 \mathrm{~dB}$ points (or another relevant attenuation level) around each notch frequency.

\paragraph{Q-Factor for Each Frequency:} Calculate the Q-factor separately for each notch frequency $(400 \mathrm{~Hz}$ and $1200 \mathrm{~Hz}$ ) based on the measured bandwidths.

\includegraphics[max width=\textwidth, center]{images/2023_12_24_70b29604aa91f5b1dce9g-4}
The provided code is for an Adaptive Notch Filter (ANF) which aims to remove two frequencies: $400 \mathrm{~Hz}$ and $1200 \mathrm{~Hz}$. However, based on the frequency response plot you've shared, it appears that only the $1200 \mathrm{~Hz}$ component is being effectively notched out.

The reason for this could be related to how the adaptive filter is designed and how it's adapting to the input signal. Here are some potential reasons why the $400 \mathrm{~Hz}$ component is not notched out:

\begin{itemize}
  \item Filter Order: The filter appears to be a second-order filter, which means it can effectively notch out one frequency at a time. If the adaptive algorithm is locking onto the strongest frequency component, which might be the $1200 \mathrm{~Hz}$ in this case, it would not properly adapt to and attenuate the $400 \mathrm{~Hz}$ component.
  \item Adaptation Speed: The adaptation speed (mu) and the parameter (rho) determine how quickly the filter coefficients adapt to the input signal. If these parameters are not set correctly, the filter might not adapt well to both frequencies, especially if they're significantly different in amplitude or if they don't alternate in a way that the adaptation algorithm can handle.
  \item Signal Composition: The way the signal is created by concatenating two sinusoids might not provide a signal that changes in a way that allows the filter to adapt to both frequencies equally. If the $1200 \mathrm{~Hz}$ component dominates when the ANF locks its coefficients, the $400 \mathrm{~Hz}$ might be effectively ignored.
  \item Filter Coefficients: The ANF uses a recursive structure with coefficients that adapt over time. If the algorithm is designed or initialized to better suit the attenuation of one frequency over another, this could lead to effective filtering of only one of the target frequencies.
  \item Frequency Response Plot: The frequency response plot you've generated is based on the final coefficient a\_i after the filter has processed the entire signal. This means it represents the state of the filter at the end of the signal processing, not throughout the process. If the filter adapted to remove $1200 \mathrm{~Hz}$ towards the end of the signal, the plot would reflect that final state.
\end{itemize}

To address the issue, you might need to:

\begin{itemize}
  \item Implement a multi-notch filter design that can handle multiple frequencies simultaneously.
  \item Adjust the adaptation parameters ( $\mathrm{mu}$ and rho) to ensure the filter can adapt to both frequencies effectively.
  \item Alter the signal generation method to ensure both frequencies are presented to the filter in a way that allows for proper adaptation.
  \item Reviewing and fine-tuning the algorithm parameters, or possibly using a different adaptive algorithm, could help in achieving better notch filtering for both frequencies. Implemented by:
  \item I. Implement a Multi-Notch Filter Design
  \item To implement a multi-notch adaptive filter capable of handling multiple frequencies, you would typically cascade multiple second-order notch filters or create a higher-order filter with multiple notches.
  \item Cascading Second-Order Filters:
  \item You can create two separate adaptive filters, each tuned to one of the frequencies you wish to remove. Then, process the signal sequentially through both filters.
  \item Higher-Order Filter with Multiple Notches:
  \item Create a higher-order filter with notches at each target frequency. This is a more complex approach and would likely require designing a custom filter or using optimization techniques to determine the filter coefficients.
  \item 
  \begin{enumerate}
    \setcounter{enumii}{1}
    \item Adjust Adaptation Parameters
  \end{enumerate}
  \item To ensure that the filter can adapt to both frequencies effectively, you might need to fine-tune the parameters mu and rho. The adaptation rate mu controls how quickly the filter coefficients adapt in response to the error signal, while rho affects the bandwidth of the notch.
  \item mu (Adaptation Rate):
  \item If $\mathrm{mu}$ is too small, the filter will adapt slowly, which might not be effective for signals with changing frequencies. If $\mathrm{mu}$ is too large, the filter might become unstable or oscillate. You need to find a balance that allows the filter to adapt quickly enough without becoming unstable.
  \item rho (Pole Radius):
  \item This parameter affects the notch bandwidth. A smaller rho results in a wider notch, which might be necessary if the frequencies you want to remove are not very stable. A larger rho creates a sharper notch, which can remove a very specific frequency but may miss nearby frequencies.
\end{itemize}

\section{What are the parameters related to $\mathbf{Q}$ factor mentioned in the literature?}

\begin{itemize}
  \item The document does mention specific parameters related to the adaptive notch filter, like the pole radius $(\rho)$ and the adaptation step size $(\mu)$, which are crucial for ensuring the stability and effectiveness of the filter. However, it does not explicitly discuss the $Q$ factor of the filter.
  \item The $\mathrm{Q}$ factor is a measure of the sharpness of the resonance peak of the filter and is indirectly related to the pole radius $(\rho)$ mentioned in the document. A higher pole radius generally leads to a higher $Q$ factor, meaning a sharper or more selective notch. In the simulations described, a pole radius of $\rho=0.9$ was found to give sufficiently narrow notches while ensuring a stable filter, which implies a relatively high $Q$ factor without specifically quantifying it
\end{itemize}

\section{What should the fix point q be for this Q-factor of ANF ( non-cascade)? The current Lower -3dB
Frequency: I 333.0 Hz, Upper -3dB Frequency: I334.0000000000002 Hz, 3dB Bandwidth:
I.0000000000002274 Hz, Q Factor: I I 99.9999999997272 ?}

\begin{itemize}
  \item The Texas Instruments TMS320C55I0 DSP is a fixed-point processor, and when dealing with very high Q factors (such as 1200), precision becomes crucial. The notation I6qI5 indicates that there is I sign bit, I6 bits in total, and 15 bits dedicated to the fractional part. This gives you a range of $\left[-1,1-2^{-15}\right]$ and a smallest step size of $2^{-15}$ (approximately 3.05e-5).
  \item Considering the high $\mathrm{Q}$ factor value you have, which is close to 1200 , and the precision required to represent small bandwidths accurately, a 16-bit representation might not be sufficient to handle the small changes in the filter coefficients or the error signal without causing quantization errors.
  \item A 32-bit fixed-point representation (e.g., 32q3I or another format with more fractional bits) would allow for a much higher precision. However, this also comes with increased computational complexity and potentially slower performance, depending on the architecture and the operations required.
  \item In a 32q3 I format, for example, you would have I sign bit and 3I bits for the fractional part, providing a smallest step size of $2^{-31}$ (approximately 4.66e-I0), which is much more precise than the $16 \mathrm{q} 15$ format.
\end{itemize}

\section{The choice of step size $\mu$ ?}

\begin{center}
\includegraphics[max width=\textwidth]{images/2023_12_24_70b29604aa91f5b1dce9g-6}
\end{center}

\section{The choice of lambda $\lambda$ ?}

\begin{center}
\includegraphics[max width=\textwidth]{images/2023_12_24_70b29604aa91f5b1dce9g-7}
\end{center}

can tms 320 c $551 \mathrm{x}$ handle $\mathrm{q} 62$ format? the register capacity is at maximum 32 right? you are correct; the TMS320C55Ix DSP series uses a 16-bit fixed-point architecture with auxiliary registers that can be combined to perform 32-bit arithmetic. It does not natively support 62-bit fixed-point arithmetic operations directly in hardware.

When two Q3I numbers are multiplied, the result is conceptually a Q62 number. However, since the DSP cannot handle 62-bit numbers directly, this operation typically results in a 32-bit result with the need for handling overflow and proper scaling to maintain the Q-format.

In the context of the TMS320C55Ix, you would use the MAC (Multiply-Accumulate) instruction, which multiplies two 16-bit numbers and accumulates the 32-bit result. When dealing with 32-bit numbers, you would perform the multiplication in parts, typically using the MPY instruction for the lower 16 bits and the MPYH instruction for the higher 16 bits, and then combining the results appropriately.

\paragraph{For example, if you need to multiply two 32-bit numbers and keep the result in Q3I format, you would:}

I. Multiply the lower 16 bits of both numbers to obtain the lower part of the result.

\begin{enumerate}
  \setcounter{enumi}{1}
  \item Multiply the higher 16 bits of both numbers to obtain the higher part of the result.

  \item Optionally, multiply the higher part of one number with the lower part of the other and add to the result, shifting as necessary.

  \item Scale back the result to fit in the 32-bit register with appropriate rounding.

\end{enumerate}

\subsection{Here's a simplified example of how this might look in the TMS320C551x assembly language:}

assembly $\bigcirc$ 

; Assuming A and B are both 32-bit Q31 format numbers stored in two reg

Multiply the lower halves and store the result

MPY AL, BL, *+ ; Multiply AL and BL, result goes to P-register

; Multiply the higher halves and add to the P-register

MPY AH, BH, PAC ; Multiply AH and BH, add to P-register with accumula

Additional cross-products and accumulations go here if needed

; Shift the P-register right by 31 to scale back to Q31 format

LSHR $P$, \#31, P

; The result in $\mathrm{P}$ is now a scaled $Q 31$ product of $\mathrm{A}$ and $\mathrm{B}$

assembly

O copy code

; Assuming A\_high, A\_low, B\_high, B\_low contain the high and low parts

; of two Q31 format numbers A and B, and you need to multiply A and B

; Multiply high parts and accumulate (high-high)

MPY A\_high, B\_high, ACO

; Multiply low of A with high of $B$ and accumulate (low-high)

MPY A\_low, B\_high, TO

LSHR TO, \#15, TO

ADD TO, ACO

; Multiply high of A with low of B and accumulate (high-low)

MPY A\_high, B\_low, TO

LSHR TO, \#15, TO

ADD TO, ACO

; Multiply low parts (low-low)

MPY A\_low, B\_low, TO

; Result in ACO is the final product in Q31 format

; Other operations like addition, subtraction, etc., should be handled

; considering the Q31 format. You might need to use saturation arithmet

; to prevent overflow.

$\downarrow$

This process involves accumulating parts of the product in the P-register and scaling the result to fit the fixed-point format used in your algorithm. It is essential to handle overflow and underflow properly, possibly using saturation arithmetic where applicable.

\section{How many bits capability is Tx, ACx, and  auxiliary register (ARx)?}
In the TMS320C55x DSP family, which includes the TMS320C5515, the size of various registers like the Auxiliary Registers (ARx), Temporary Registers (Tx), and Accumulators (ACx) are as follows:
\subsection{Auxiliary Registers (ARx)}
The ARx registers (AR0, AR1, AR2, AR3, AR4, AR5, AR6, AR7) are 16-bit registers. They are generally used for addressing and can be combined to form 32-bit addresses in certain addressing modes.
\subsection{Temporary Registers (Tx)} The Tx registers (T0, T1, T2, T3) are 16-bit registers. They are used for temporary storage of data during calculations.
\subsection{Accumulators (ACx)} The accumulators (AC0, AC1, AC2, AC3) in the TMS320C55x DSPs are 40-bit registers. Each accumulator is typically divided into a 16-bit high part (referred to as ACH) and a 24-bit low part (referred to as ACL). This 40-bit structure allows for extended precision and helps in managing the overflow in arithmetic calculations, which is especially important in fixed-point arithmetic commonly used in DSP applications.

The extended size of the accumulators is particularly useful in digital signal processing where products of 16-bit numbers are common. \textbf{The 40-bit accumulator allows for the accumulation of these products without overflow}, thus maintaining precision in calculations.
